\documentclass[a4paper, 11pt, dvipdfmx]{jsarticle}

% 構造式
\usepackage{chemfig}
\setchemfig{atom sep=2em}

\begin{document}

  \section{1}

    \begin{figure}[htbp]
      \begin{tabular}{cc}
        \counterwithin{figure}{section}

        \begin{minipage}[t]{0.33\textwidth}
          \centering
          \chemfig{
              -C(=[-2]O)-
            }
          \caption{ケトン基}
        \end{minipage}

        \begin{minipage}[t]{0.33\textwidth}
          \centering
          \chemfig{
              -C(=[-2]O)-H
            }
          \caption{ホルミル基}
        \end{minipage}

        \begin{minipage}[t]{0.33\textwidth}
          \centering
          \chemfig{
              -N(-[2]H)-C(=[-2]O)-
            }
          \caption{アミド結合}
        \end{minipage}

      \end{tabular}
    \end{figure}

    \begin{figure}[htbp]
      \begin{tabular}{cc}

        \begin{minipage}[t]{0.33\textwidth}
          \centering
          \chemfig{
              -C(=[-2]O)-OH
            }
          \caption{カルボキシ基}
        \end{minipage}

      \end{tabular}
    \end{figure}

  \section{2}

    \begin{figure}[htbp]
      \begin{tabular}{cc}
        \counterwithin{figure}{section}

        \begin{minipage}[t]{0.33\textwidth}
          \centering
          \chemfig{
              CH_3-C(-[2]CH_3)(-[-2]CH_3)-CH_2-CH_2-CH_3
            }
          \caption{2.2-ジメチルペンタン}
        \end{minipage}

        \begin{minipage}[t]{0.33\textwidth}
          \centering
          \chemfig{
              C(-[3]CH_3)(-[5]H)=C(-[1]CH_3)(-[-1]H)
            }
          \caption{シス-2-ブテン}
        \end{minipage}

        \begin{minipage}[t]{0.33\textwidth}
          \centering
          \chemfig{
              CH_3-C(-[-2]CH_3)=CH-CH_3
            }
          \caption{2-メチル-2-ブテン}
        \end{minipage}

      \end{tabular}
    \end{figure}



    \begin{figure}[htbp]
      \begin{tabular}{cc}

        \begin{minipage}[t]{0.33\textwidth}
          \centering
          \chemfig{
              CH_2(-[-2]OH)-CH_2(-[-2]OH)
            }
          \caption{エチレングリコール}
        \end{minipage}

        \begin{minipage}[t]{0.33\textwidth}
          \centering
          \chemfig{
              C(=[-2]O)(-[3]H)(-[1]H)
            }
          \caption{ホルムアルデヒド}
        \end{minipage}

        \begin{minipage}[t]{0.33\textwidth}
          \centering
          \chemfig{
              C(-[3]H)(-[5]C(=[-2]O)-[4]HO)=C(-[1]H)(-[-1]C(=[-2]O)-OH)
            }
          \caption{マレイン酸}
        \end{minipage}

      \end{tabular}
    \end{figure}
    


    \begin{figure}[htbp]

      \begin{tabular}{cc}

        \begin{minipage}[t]{0.33\textwidth}
          \centering
          \chemfig{
              CH_3-CH_2-O-CH(=[-2]O)
            }
          \caption{ギ酸エチル}
        \end{minipage}

        \begin{minipage}[t]{0.33\textwidth}
          \centering
          \chemfig{
              C(-[3]H)(-[5]H)=C(-[1]H)(-[-1]Cl)
            }
          \caption{塩化ビニル}
        \end{minipage}

        \begin{minipage}[t]{0.33\textwidth}
          \centering
          \chemfig{
            CH_3-CH_2-^{*}C(-[2,,2]Cl)(-[-2,,2]H)-CH_3
          }
          \caption{1-クロロ-2-メチルブタン}
        \end{minipage}

      \end{tabular}
    \end{figure}

  \section{3}

    \begin{figure}[htbp]
      \begin{tabular}{cc}
        \counterwithin{figure}{section}

        \begin{minipage}[t]{0.5\textwidth}
          \centering
          \chemfig{
              CH_3-[,0.8]CH_2-[,0.8]CH(-[2, 0.8]CH_2-[2, 0.8]CH_3)-[,0.8]CH_2-[,0.8]CH_2-[,0.8]CH_3
            }
          \caption{}
        \end{minipage}

        \begin{minipage}[t]{0.5\textwidth}
          \centering
          \chemfig{
              CH_3-[,0.8]CH_2-[,0.8]CH(-[2, 0.8]CH_3)-[,0.8]C(-[2, 0.8]CH_3)(-[-2, 0.8]CH_3)-[,0.8]CH_3
            }
          \caption{}
        \end{minipage}

      \end{tabular}
    \end{figure}

    \begin{figure}[htbp]
      \begin{tabular}{cc}

        \begin{minipage}[t]{0.33\textwidth}
          \centering
          \chemfig{
              C(-[3]H)(-[5]CH_3)=C(-[1]CH_3)(-[-1]H)
            }
          \caption{}
        \end{minipage}

        \begin{minipage}[t]{0.33\textwidth}
          \centering
          \chemfig{
              CH_3-CH_2-O-CH_2-CH_3
            }
          \caption{}
        \end{minipage}

      \end{tabular}
    \end{figure}

    \begin{figure}[htbp]
      \begin{tabular}{cc}

        \begin{minipage}[t]{0.33\textwidth}
          \centering
          \chemfig{
              C(-[3]H)(-[5]H)=C(-[1]H)(-[-1]O-C(=[-2]O)-CH_3)
            }
          \caption{}
        \end{minipage}

        \begin{minipage}[t]{0.33\textwidth}
          \centering
          \chemfig{
              C(-[3]H)(-[5]H)=C(-[1]H)(-[-1]CN)
            }
          \caption{}
        \end{minipage}

        \begin{minipage}[t]{0.33\textwidth}
          \centering
          \chemfig{
              CH_3-C(=[-2]O)-CH_3
            }
          \caption{}
        \end{minipage}

      \end{tabular}
    \end{figure}

    \begin{figure}[htbp]
      \begin{tabular}{cc}

        \begin{minipage}[t]{0.33\textwidth}
          \centering
          \chemfig{
              H-C(=[-2]O)-OH
            }
          \caption{}
        \end{minipage}

        \begin{minipage}[t]{0.33\textwidth}
          \centering
          \chemfig{
              H-C(-[2]H)(-[-2]Cl)-C(-[2]H)(-[-2]Cl)-H
            }
          \caption{}
        \end{minipage}

        \begin{minipage}[t]{0.33\textwidth}
          \centering
          \chemfig{
              C(-[3]H)(-[5]H)=C(-[1]H)(-[-1]OH)
            }
          \caption{}
        \end{minipage}

      \end{tabular}
    \end{figure}

    \begin{figure}
      \begin{tabular}{cc}

        \begin{minipage}[t]{0.33\textwidth}
          \centering
          \chemfig{
              C(=[-2]O)(-[3]CH_3)(-[1]H)
            }
          \caption{アセトアルデヒド}
        \end{minipage}

      \end{tabular}
    \end{figure}

    \section{4}

    \begin{figure}

      \begin{tabular}{cc}
        \centering
        \begin{minipage}[t]{0.33\textwidth}
          \centering
          \chemfig{
              % C-C-C-C-C-C-C
              [-1]-[1]-[-1]-[1]-[-1]-[1]-[-1]
            }
        \end{minipage}

        \begin{minipage}[t]{0.33\textwidth}
          \centering
          \chemfig{
              C-C(-[-2]C)-C-C-C-C
            }
        \end{minipage}

        \begin{minipage}[t]{0.33\textwidth}
          \centering
          \chemfig{
              C-C-C(-[-2]C)-C-C-C
            }
        \end{minipage}

      \end{tabular}

      \begin{tabular}{cc}
        \centering

        \begin{minipage}[t]{0.33\textwidth}
          \centering
          \chemfig{
              C-C(-[2]C)-C(-[-2]C)-C-C
            }
        \end{minipage}

        \begin{minipage}[t]{0.33\textwidth}
          \centering
          \chemfig{
              C-C(-[2]C)-C-C(-[-2]C)-C
            }
        \end{minipage}

        \begin{minipage}[t]{0.33\textwidth}
          \centering
          \chemfig{
              C-C-C(-[2]C)(-[-2]C)-C-C
            }
        \end{minipage}

      \end{tabular}

      \begin{tabular}{cc}
        \centering

        \begin{minipage}[t]{0.33\textwidth}
          \centering
          \chemfig{
              C-C(-[-2]C)(-[2]C)-C-C-C
            }
        \end{minipage}

        \begin{minipage}[t]{0.33\textwidth}
          \centering
          \chemfig{
              C-C-C(-[-2]C(-[-2]C))-C-C
            }
        \end{minipage}

        \begin{minipage}[t]{0.33\textwidth}
          \centering
          \chemfig{
              C-C(-[-2]C)-C(-[2]C)(-[-2]C)-C
            }
        \end{minipage}

      \end{tabular}
      
      \caption{}
    \end{figure}

    \begin{figure}

      \begin{tabular}{cc}
        \centering
        \begin{minipage}[t]{0.33\textwidth}
          \centering
          \chemfig{
              C=C-C-C-C
            }
        \end{minipage}

        \begin{minipage}[t]{0.33\textwidth}
          \centering
          \chemfig{
              C-C=C-C-C
            }
        \end{minipage}

        \begin{minipage}[t]{0.33\textwidth}
          \centering
          \chemfig{
              C=C(-[-2]C)-C-C
            }
        \end{minipage}

      \end{tabular}

      \begin{tabular}{cc}
        \centering

        \begin{minipage}[t]{0.33\textwidth}
          \centering
          \chemfig{
              C=C-C(-[-2]C)-C
            }
        \end{minipage}

        \begin{minipage}[t]{0.33\textwidth}
          \centering
          \chemfig{
              C-C(-[-2]C)=C-C
            }
        \end{minipage}

        \begin{minipage}[t]{0.33\textwidth}
          \centering
          \chemfig{
              [:18]*5(C?-C-C-C-C?)
            }
        \end{minipage}

      \end{tabular}

      \begin{tabular}{cc}
        \centering

        \begin{minipage}[t]{0.33\textwidth}
          \centering
          \chemfig{
              *4(C?-C(-[0]C)-C-C?)
            }
        \end{minipage}

        \begin{minipage}[t]{0.33\textwidth}
          \centering
          \chemfig{
              [:-30]*3(C?-C(-[0]C-[0]C)-C?)
            }
        \end{minipage}

        \begin{minipage}[t]{0.33\textwidth}
          \centering
          \chemfig{
              [:-30]*3(C?-C(-[0]C)-C?(-[0]C))
            }
        \end{minipage}

      \end{tabular}

      \begin{tabular}{cc}
        \centering
        \begin{minipage}[t]{1\textwidth}
          \centering
          \chemfig{
              [:-30]*3(C?-C(-[0]C)(-[-1]C)-C?)
            }
        \end{minipage}
      \end{tabular}

      \caption{}
    \end{figure}



    \begin{figure}

      \begin{tabular}{cc}
        \centering
        \begin{minipage}[t]{0.33\textwidth}
          \centering
          \chemfig{
              1-C-^{*}C(-[2,,2]2)-^{*}C(-[2,,2]3)-C-C-C
            }
        \end{minipage}

        \begin{minipage}[t]{0.33\textwidth}
          \centering
          \chemfig{
              ^{*}C(-[-2,,2]5)(-[3]C)(-[5]C-[4]4)-^{*}C(-[-2,,2]6)-^{*}C(-[-2,,2]7)-C(-[-2]8)
            }
        \end{minipage}

        \begin{minipage}[t]{0.33\textwidth}
          \centering
          \chemfig{
              9-C-^{*}C(-[-2,,2]10)-^{*}C(-[2,,2]C(-[2]12))(-[-2,,2]11)-C-C
            }
        \end{minipage}

      \end{tabular}

      \begin{tabular}{cc}
        \centering
        \begin{minipage}[t]{0.5\textwidth}
          \centering
          \chemfig{
              ^{*}C(-[-2,,2]14)(-[3]C)(-[5]C-[4]13)-C(-[1]C)(-[-1]C)
            }
        \end{minipage}

        \begin{minipage}[t]{0.5\textwidth}
          \centering
          \chemfig{
              15-C-C(-[2]C)(-[-2]C)-^{*}C(-[-2,,2]16)-C(-[-2]17)
            }
        \end{minipage}

      \end{tabular}

      \caption{}
    \end{figure}


    \begin{figure}

      \begin{tabular}{cc}
        \centering
        \begin{minipage}[t]{0.33\textwidth}
          \centering
          \chemfig{
              Cl-C(-[-2]1)-C(-[-2]2)-C(-[-2]3)-C(-[-2]4)
            }
        \end{minipage}

        \begin{minipage}[t]{0.33\textwidth}
          \centering
          \chemfig{
              C-C(-[2]Cl)(-[-2]5)-C(-[-2]6)-C
            }
        \end{minipage}

        \begin{minipage}[t]{0.33\textwidth}
          \centering
          \chemfig{
              Cl-C(-[-2]7)-C(-[-2]8)(-[2]C)-C(-[-2]9)
            }
        \end{minipage}

      \end{tabular}

      \caption{}
    \end{figure}

\end{document}